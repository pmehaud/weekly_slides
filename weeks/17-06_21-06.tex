\smallframetitle

\section{From 17/06/24 to 21/06/24}
\insertsectionframe

\subsection{Road detection method - in progress}
\insertsubsectionframe

\begin{frame}{Methodology}
    We have seen that one classification with either DBScan or HDBScan isn't enough. So, why not combining them.
    \begin{block}{The base idea}
        We will combine the clusters of DBScan, HDBScan and OPTICS (another density base clustering method) to try to find roads.
    \end{block}

    \begin{block}{One problem}
        We are now able to detect road/railways but the problem is that there is still little clusters that are useless.
    \end{block}
\end{frame}

\begin{frame}{DBScan clustering}
    \begin{figure}
        \includegraphics[height=0.6\paperheight]{images/cartes/road_detection/dbs.png}
        \caption{DBScan clustering}
    \end{figure}
\end{frame}

\begin{frame}{HDBScan clustering}
    \begin{figure}
        \includegraphics[height=0.6\paperheight]{images/cartes/road_detection/hdb.png}
        \caption{HDBScan clustering}
    \end{figure}
\end{frame}

\begin{frame}{OPTICS clustering}
    \begin{figure}
        \includegraphics[height=0.6\paperheight]{images/cartes/road_detection/opt.png}
        \caption{OPTICS clustering}
    \end{figure}
\end{frame}

\begin{frame}{Result}
    \begin{figure}
        \includegraphics[height=0.6\paperheight]{images/cartes/road_detection/res.png}
        \caption{Road detection}
    \end{figure}
\end{frame}

\subsection{More details and problems}
\insertsubsectionframe

\begin{frame}{Detailed clusters detected}
    \begin{figure}
        \includegraphics[height=0.6\paperheight]{images/clusters_road_detection.html.png}
        \caption{Detailed clusters detected in coutryside by each method}
    \end{figure}
\end{frame}

\begin{frame}{Problems and ideas}
    \begin{block}{Problems}
        \begin{itemize}
            \item A lot of little citys detected : not only roads ;
            \item The areas around big citys are a mess ;
            \item HDBScan has only one cluster.
        \end{itemize}
    \end{block}

    \begin{block}{Ideas}
        \begin{itemize}
            \item Refine the parameters of each method ;
            \item Use linear regressions to detect parts of roads and maybe help propagate them.
        \end{itemize}
    \end{block}
\end{frame}

\subsection{Altair AI Studio Software Review}
\insertsubsectionframe

\begin{frame}{Altair AI Studio Software Review}
    Objective is to evaluate Altair AI Studio for their potential to enhance our research on classifying the terrain of mobile base stations.
    Available at: \url{https://altair.com/altair-ai-studio}
    \begin{block}{Altair AI Studio}
        This is a platform designed for data analysis and machine learning model building. Possible benefits for us:
        \begin{itemize}
            \item Supports clustering and classification algorithms.
            \item Support for various machine learning algorithms.
            \item Enables effective result visualization (graphs) for enhanced analysis.
        \end{itemize}
    \end{block}
    \begin{columns}
        \begin{column}{0.4\paperwidth}
            \begin{block}{Key Features:}
                \begin{itemize}
                    \item Integration with various data sources.
                    \item Interactive model creation and testing.
                    \item Support for various machine learning algorithms.
                \end{itemize}
            \end{block}
        \end{column}
        \begin{column}{0.4\paperwidth}
            \begin{block}{Users:}
                \begin{itemize}
                    \item Data researchers
                    \item Analysts
                    \item Machine learning developers
                \end{itemize}
            \end{block}
        \end{column}
    \end{columns}
\end{frame}

\begin{frame}{Example of Usage in Altair AI Studio}
    \begin{figure}
        \includegraphics[height=0.6\paperheight]{images/Altair/Altair_proc_exmpl.png}
        \caption{How Altair AI Software works}
    \end{figure}
\end{frame}


\subsection{New Possible Approach for Terrain Classification}
\insertsubsectionframe

\begin{frame}{More accurate verification of the classification of base station locations}
We want to validate and enhance the previous classification of base station locations by leveraging a new geographic dataset. 
We found and will try to use a comprehensive dataset from \url{data.enseignementsup-recherche.gouv.fr}, which includes precise geographic data points across France with pre-defined zone classifications.
\begin{block}{Methodology for Base Station Classification}
    Previously, we worked with a dataset containing locations of all base stations. Now, we utilize an additional dataset with geographic points across France, each classified into zones (urban, suburban, rural). By comparing base station coordinates with these geographic points, we determine the zone of each base station. The closest geographic point's classification is assigned to the base station, providing a more accurate terrain classification.
\end{block}
    \begin{block}{Benefits:}
        \begin{itemize}
            \item Cross-reference base station coordinates with geographic data points to accurately assign zone classifications.
            \item Provides an additional layer of verification and accuracy for our classification methods.
        \end{itemize}
    \end{block}
\end{frame}



\begin{frame}{Classification of points in the new dataset}
    \begin{figure}
        \includegraphics[height=0.6\paperheight]{images/Geo_approach/New_dataset_points_ill.png}
        \caption{Classification of points from the new dataset}
    \end{figure}
\end{frame}
\begin{frame}{Classification of points in the new dataset}
    \begin{block}{Geographic identifiers:}
        \begin{itemize}
        \item UU (Urban Units): Areas classified as urban units.
        \item CR (Rural Communes): Areas classified as rural communes.
        \item AU (Urban Areas): Larger urban areas or agglomerations.
        \end{itemize}
    \end{block}
    \begin{block}{Methodology and Challenges}
            This method helps classify base station locations by comparing them with geographic data points. However, we need to preprocess the dataset carefully and address the issue of having a limited number of points. To ensure high accuracy, we might need to augment the data or combine it with other datasets for better coverage and precision
    \end{block}
\end{frame}