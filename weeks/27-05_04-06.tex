\smallframetitle

\section{Semaine du 21/05/24 au 27/05/24}
\insertsectionframe

\subsection{Amélioration des critères de sélection}
\insertsubsectionframe

\begin{frame}{Prise en compte de la probabilité d'être dans une ville}
    \begin{block}{Concept}
        Chaque station possède une probabilité d'être dans une ville.
        On va donc utiliser ce paramètre afin de moduler l'angle et la distance minimale entre 2 stations.
    \end{block}

    \begin{block}{Choix d'implantation}
        Pour l'instant, on a séparé les stations en 4 catégories, en fonction de la probabilité d'être une ville:
        \begin{itemize}
            \item $p=1$ : \texttt{distance\_max = 1} et \texttt{min\_angle = 45};
            \item $p=0$ : \texttt{distance\_max = 15} et \texttt{min\_angle = 15};
            \item $p\in\left]1 ; 0,6\right[$ : \texttt{distance\_max = 5} et \texttt{min\_angle = 30};
            \item $p\in\left[0,6 ; 0\right[$ : \texttt{distance\_max = 10} et \texttt{min\_angle = 20}.
        \end{itemize}
    \end{block}
    
    \begin{alertblock}{Pistes de travail}
        Il va maintenant falloir effectuer des expérimentations pour trouver les bons paramètres et peut-être essayer de savoir quel paramètre est le plus pertinent.
    \end{alertblock}
\end{frame}

\begin{frame}{Améliorations des cadrants}
    Dans un premier temps, nous avons décidé de se laisser la possibilité de choisir plus d'un voisin par cadrant.
    Cette option ne sera peut-être pas gardée car elle n'élimine pas assez de voisins potentiels.
\end{frame}