\smallframetitle

\section{From 01/07/24 to 05/07/24}
\insertsectionframe


    
\begin{frame}{Creation of New Database}
    \begin{itemize}
        \item We have created a new comprehensive database containing detailed information about base stations and their antennas in Normandy, France.
    \end{itemize}
    \begin{block}{Database Fields}
        \begin{itemize}
            \item \textbf{id\_station\_anfr}: Identifier for the station
            \item \textbf{latitude}: Latitude coordinate
            \item \textbf{longitude}: Longitude coordinate
            \item \textbf{nom\_dep}: Department name
            \item \textbf{nom\_com}: Commune name
            \item \textbf{site\_2g}: Presence of 2G technology (1 = Yes, 0 = No)
            \item \textbf{site\_3g}: Presence of 3G technology (1 = Yes, 0 = No)
            \item \textbf{site\_4g}: Presence of 4G technology (1 = Yes, 0 = No)
            \item \textbf{site\_5g}: Presence of 5G technology (1 = Yes, 0 = No)
            \item \textbf{AER\_NB\_DIMENSION}: Dimension of the antenna
            \item \textbf{AER\_NB\_AZIMUT}: Azimuth angle of the antenna
        \end{itemize}
    \end{block}
\end{frame}
    
\begin{frame}{Program for Visualization}
    \begin{itemize}
        \item We implemented a Python program to visualize the base stations on a map and analyze their coverage using Voronoi diagrams.
    \end{itemize}
    \begin{block}{Key Steps in the Program}
        \begin{itemize}
            \item Load and clean the data.
            \item Create a Folium map.
            \item Add markers and pop-ups for each base station.
            \item Calculate and visualize Voronoi cells and antenna sectors.
        \end{itemize}
    \end{block}
\end{frame}
    
\begin{frame}{Detailed Steps in the Visualization Process}
    \begin{block}{Step 1: Load and Clean Data}
        \begin{itemize}
            \item Load the dataset using pandas.
            \item Clean numeric columns to ensure proper data types.
        \end{itemize}
    \end{block}
    \begin{block}{Step 2: Filter Data for Normandy}
        \begin{itemize}
            \item Filter the dataset to include only entries from Normandy departments.
        \end{itemize}
    \end{block}
    \begin{block}{Step 3: Create and Center the Map}
        \begin{itemize}
            \item Calculate the map center based on average coordinates.
            \item Initialize a Folium map centered on this location.
        \end{itemize}
    \end{block}
\end{frame}
    
\begin{frame}{Adding Markers and Voronoi Cells}
    \begin{block}{Step 4: Add Markers for Each Base Station}
        \begin{itemize}
            \item Group data by station ID.
            \item Add markers to the map with pop-ups displaying detailed information.
        \end{itemize}
    \end{block}
    \begin{block}{Step 5: Calculate and Visualize Voronoi Cells}
        \begin{itemize}
            \item Perform Voronoi tessellation on the coordinates.
            \item Define cell borders based on antenna azimuths and draw them on the map.
        \end{itemize}
    \end{block}
\end{frame}
    
\begin{frame}{Antenna Coverage and Sector Visualization}
    \begin{block}{Step 6: Draw Antenna Sectors}
        \begin{itemize}
            \item Calculate intersections with Voronoi cell boundaries based on antenna azimuths.
            \item Draw sectors on the map to visualize antenna coverage.
        \end{itemize}
    \end{block}
    \begin{block}{Step 7: Save and Display the Map}
        \begin{itemize}
            \item Add a layer control to toggle between markers and Voronoi cells.
            \item Save the final map to an HTML file for interactive viewing.
        \end{itemize}
    \end{block}
\end{frame}

\begin{frame}{Conclusion}
    \begin{block}{Summary}
        \begin{itemize}
            \item We have successfully created a new database and visualized the base stations and their coverage areas in Normandy.
            \item The Voronoi diagram and sector visualization provide a detailed view of the antenna coverage.
        \end{itemize}
    \end{block}
    \begin{block}{Future Work}
        \begin{itemize}
            \item Enhance the visualization with additional data layers.
            \item Use the visualized data for further analysis and network optimization.
        \end{itemize}
    \end{block}
\end{frame}

\end{document}
