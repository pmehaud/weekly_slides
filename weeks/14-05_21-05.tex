\smallframetitle

\section{Semaine du 14/05/24 au 21/05/24}
\insertsectionframe
\subsection{Le jeu de donnée}
\insertsubsectionframe

\begin{frame}{Le jeu de donnée}
    \begin{block}{Arcep}
        L'autorité de régulation des communications électroniques, des postes et de la distribution de la presse (Arcep) est une autorité administrative indépendante française chargée de réguler les communications électroniques et postales et la distribution de la presse.
    \end{block}

    \begin{block}{Mon Réseau Mobile}
        Mon Réseau Mobile est la plate-forme cartographique regroupant l’ensemble des données géographiques en lien avec les réseaux dits \og mobiles \fg{} (2G, 3G, 4G, 5G) régulés par l’Arcep.
    \end{block}
\end{frame}


\begin{frame}{Arborescence du jeu de données}
    \begin{columns}
        \begin{column}{0.4\textwidth}
            Les fichiers de données sont rangés par trimestre de publication, zone (France métropolitaine/Outre-mer) et département le cas échéant :
        \end{column}
            
        \begin{column}{0.6\textwidth}
            \begin{figure}
                \includegraphics[height=0.55\paperheight]{images/architecture.png}
                \caption{\label{fig:archi}Architecture de la base de donnée}
            \end{figure}
        \end{column}
    \end{columns} 
    
\end{frame}

\begin{frame}{Fréquences de mises à jour}
    \begin{figure}
        \includegraphics[height=0.5\paperheight]{images/frequence.png}
        \caption{\label{fig:freq}Fréquence de publication de mises à jour}
    \end{figure}
\end{frame}


\subsection{Faisons parler les données}
\insertsubsectionframe



\begin{frame}{Comparaison des différents équipements en terme de technologies (1/7)}
    Voici tout d'abord un graphique sur la présence d'une technologie en fonction de l'opérateur (une technologie présente sur un site n'exclue pas la présence d'une autre technologie) :
    \begin{figure}
        \includegraphics[height=0.4\paperheight]{images/barplots/avec_techno.png}
        \caption{\label{fig:av_tech}Nombres de sites équipés d'au moins une technologie}
    \end{figure}
        
\end{frame}

\begin{frame}{Comparaison des différents équipements en terme de technologies (2/7)}
    \begin{table}[!ht]
        \centering
        \footnotesize
        \begin{tabular}{cccccc}
        \hline
            \textbf{Technologie} & \textbf{Bouygues Telecom} & \textbf{Free Mobile} & \textbf{Orange} & \textbf{SFR} & \textbf{Total} \\ \hline
            \textbf{2G} & 4 & 0 & 10 & 20 & 34 \\ 
            \textbf{3G} & 30 & 102 & 65 & 37 & 234 \\ 
            \textbf{4G} & 15 & 15 & 602 & 106 & 738 \\ 
            \textbf{5G} & 0 & 0 & 3 & 0 & 3 \\ 
            \textbf{2-3G} & 66 & 0 & 21 & 80 & 167 \\ 
            \textbf{2-4G} & 12 & 0 & 63 & 67 & 142 \\ 
            \textbf{2-5G} & 0 & 0 & 0 & 0 & 0 \\ 
            \textbf{3-4G} & 3889 & 7225 & 9288 & 4909 & 25311 \\ 
            \textbf{3-5G} & 0 & 0 & 0 & 0 & 0 \\ 
            \textbf{4-5G} & 6 & 0 & 109 & 38 & 153 \\ 
            \textbf{2-3-4G} & 11044 & 0 & 11697 & 9831 & 32572 \\ 
            \textbf{2-3-5G} & 0 & 0 & 1 & 0 & 1 \\ 
            \textbf{2-4-5G} & 0 & 0 & 5 & 10 & 15 \\ 
            \textbf{3-4-5G} & 681 & 18607 & 1638 & 835 & 21761 \\ 
            \textbf{2-3-4-5G} & 10584 & 0 & 7038 & 10085 & 27707 \\ 
            \textbf{Total} & 26331 & 25949 & 30540 & 26018 & 108838 \\ \hline
        \end{tabular}
        \caption{Résumé des données de présence de technologie}
    \end{table}
    
\end{frame}

\begin{frame}{Comparaison des différents équipements en terme de technologies (3/7)}
    Maintenant nous nous intéressons à la fréquence de présence de certaines technologies et pas d'autres :
    \begin{figure}
        \includegraphics[height=0.4\paperheight]{images/barplots/xG.png}
        \caption{\label{fig:xG}Nombres de sites équipés d'une unique technologie}
    \end{figure}
    
\end{frame}

\begin{frame}{Comparaison des différents équipements en terme de technologies (4/7)}
    \begin{figure}
        \includegraphics[height=0.4\paperheight]{images/barplots/2-xG.png}
        \caption{\label{fig:2-xG}Nombres de sites équipés de deux technologies}
    \end{figure}
\end{frame}

\begin{frame}{Comparaison des différents équipements en terme de technologies (5/7)}
    \begin{columns}
        \begin{column}{0.65\textwidth}
            \begin{figure}
                \includegraphics[height=0.4\paperheight]{images/barplots/3-xG.png}
                \caption{\label{fig:3-xG}Nombres de sites équipés de deux technologies (suite)}
            \end{figure}
        \end{column}
            
        \begin{column}{0.35\textwidth}
            \begin{figure}
                \includegraphics[height=0.4\paperheight]{images/barplots/4-5G.png}
                \caption{\label{fig:4-5G}Nombres de sites équipés de deux technologies (suite-bis)}
            \end{figure}
        \end{column}
    \end{columns} 
\end{frame}

\begin{frame}{Comparaison des différents équipements en terme de technologies (6/7)}
    \begin{figure}
        \includegraphics[height=0.4\paperheight]{images/barplots/2-x-yG.png}
        \caption{\label{fig:2-x-yG}Nombres de sites équipés de trois technologies}
    \end{figure}
\end{frame}

\begin{frame}{Comparaison des différents équipements en terme de technologies (7/7)}
    \begin{columns}
        \begin{column}{0.35\textwidth}
            \begin{figure}
                \includegraphics[height=0.4\paperheight]{images/barplots/3-4-5G.png}
                \caption{\label{fig:3-4-5G}Nombres de sites équipés de trois technologies (suite)}
            \end{figure}
        \end{column}
            
        \begin{column}{0.65\textwidth}
            \begin{figure}
                \includegraphics[height=0.4\paperheight]{images/barplots/all-tot.png}
                \caption{\label{fig:all-tot}Nombre de sites équipés de toutes les technologies et total}
            \end{figure}
        \end{column}
    \end{columns} 
\end{frame}

\subsection{Affichage plus détaillé des cartes}
\insertsubsectionframe

\subsubsection{Les stations de base par opérateurs}
\begin{frame}{Les stations de base par opérateur}
    \begin{figure}
        \includegraphics[width=0.9\paperheight]{images/cartes/subplots-providers.png}
        \caption{\label{fig:sp-prov}Les stations de base par opérateurs}
    \end{figure}
\end{frame}


\subsubsection{Les technologies par opérateurs}
\subsubsection{Les technologies par opérateurs}
\begin{frame}{Les stations 2G}
    \begin{figure}
        \includegraphics[width=0.9\paperheight]{images/cartes/site_2g.png}
        \caption{\label{fig:sp-2g}Les stations 2G}
    \end{figure}
\end{frame}

\begin{frame}{Les stations 3G}
    \begin{figure}
        \includegraphics[width=0.9\paperheight]{images/cartes/site_3g.png}
        \caption{\label{fig:sp-3g}Les stations 3G}
    \end{figure}
\end{frame}

\begin{frame}{Les stations 4G}
    \begin{figure}
        \includegraphics[width=0.9\paperheight]{images/cartes/site_4g.png}
        \caption{\label{fig:sp-4g}Les stations 4G}
    \end{figure}
\end{frame}

\begin{frame}{Les stations 5G}
    \begin{figure}
        \includegraphics[width=0.9\paperheight]{images/cartes/site_5g.png}
        \caption{\label{fig:sp-5g}Les stations 5G}
    \end{figure}
\end{frame}


\subsection{Zoom sur la 5g}
\insertsubsectionframe

\begin{frame}{Evolution du nombre de stations 5G en France}
    \begin{figure}
        \includegraphics[height=0.5\paperheight]{images/5g-evolution.png}
        \caption{\label{fig:5g-ev}Evolution du nombre de stations 5G en France}
    \end{figure}
\end{frame}



